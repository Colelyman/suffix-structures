\documentclass{bmc_article}

\title{Suffix tree implementation for GNUMAP}

\author{Cole Lyman\\
\small Computational Science Lab\\
\small Department of Computer Science, Brigham Young University, Provo, UT 84602 USA\\
}

\date{\today}

\begin{document}
\maketitle

\begin{abstract}
\textbf{Summary:} GNUMAP\cite{gnumap} is a genome mapper built for next-generation sequencing. 
The purpose of this paper is to improve the speed of GNUMAP by replacing the hash table currently 
used in it with a suffix tree. This will hopefully optimize GNUMAP's runtime.

\textbf{Availability:} GNUMAP can be downloaded from http://dna.cs.byu.edu/gnumap.

\textbf{Contact:} cole@colelyman.com
\end{abstract}

\section{Introduction}

\section{Tool Description}

\section{Performance Analysis}
\subsection{Comparison of Hash Table and Suffix Tree Independently} 
Using the implementation of the hash table present in GNUMAP and the suffix tree implementation
provided for GNUMAP, a comparison will be made between the two data structures independent of 
GNUMAP. 

\section{Conclusion}

\renewcommand{\abstractname}{Acknowledgements}
\begin{abstract}
The authors would like to thank each of the members of the Computational Sciences Lab at Brigham Young University for all of
their support, feedback, and guidance on this project.
\end{abstract} 

\bibliographystyle{plain}
\bibliography{./bib/gnumap,./bib/ukkonen}
\end{document}
